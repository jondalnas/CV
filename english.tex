\section{Professional Competences}
	I have a bachelor's degree in Electrical engineering and a master's degree in Computer Science, from the Technical University of Denmark.
	I am interested in the interplay between hardware and software design, in optimizing and debugging existing code, and in exploring the Linux operating system.
	I have my whole life been interested in programming and debugging, when, at the age of 12, I began my programming adventure in Java and soon after played with the Arduino UNO.
	Ever since then, I have had an enormous interest in exploring new avenues, where a lot of my free time is spent on smaller hobby programming projects in different languages, and maintaining my own Linux server.
	I am always ready for a challenge, if it means I can learn something new, and I am always open to tackling new and interesting challenges, with a drive for exploration.

\section{Work Experiences}
	\begin{itemize}
		\item Feb 2025-Now: Treasurer at K-Net
		\item Jun 2024-Now: Daglig Driftsleder at Nybrogård Netgroup
		\item Jan 2023-Now: 2024: Administrator at Nybrogård Netgroup
		\item Sep 2024-Dec 2024: Teachers assistant in 34365 IoT Prototyping at DTU
		\item Jan 2024-Jun 2024: Teachers assistant in 34373 Introduction to microcontroller development for IoT using embedded C at DTU
	\end{itemize}

\section{Education}
	\begin{itemize}
		\item 2023-2025: Master of Science in Engineering Computer Science and Engineering, DTU
		\item 2020-2023: Bachelor of Science in Engineering Electrical Engineering, DTU
		\item 2017-2020: Higher technical exam (HTX), H.C. Ørsted Gymnasiet Lyngby
	\end{itemize}

\section{Projects}
	\begin{itemize}
		\item \textbf{Bachelor's Project - \textit{Implementation of a nonlinear differential equation on a Raspberry Pi including visualization}}

			The project was about developing script to solve nonlinear differential equations on a Raspberry Pi.
			Over the course of the project, I learned Python, to optimize code algorithmically, and to work independently on a project as a co-leader.

		\item \textbf{Master's Thesis - \textit{Development and implementation of embedded low-power tracking platform}}

			The project was about creating a low-power tracker, which should be able to run off a single CR2032 battery for 10 years.
			Over the course of the project, I learned to develop low-power code, to create scripts used for visualizing different measurements, to design PCB's, and to develop universal C code utilizing defines and ifdef.

		\item \textbf{Special Course - \textit{Hardware accelerators for AI-based applications}}

			The project was about designing modules for an AI in Chisel based on the ONNX standard, then synthesizing it on an FPGA.
			Over the course  of the project, I learned the ONNX standard for AI, to work in a group designing hardware, and to complete a project without the need for guidance.

		\item \textbf{Hobby Projects}
			
			On my \href{https://github.com/jondalnas}{GitHub}, most of my personal projects can be found.
			A tour of the most interesting repositories can be found at my \href{https://github.com/jondalnas/CV}{CV}.
	\end{itemize}


\section{Languages}
	\begin{itemize}
		\item Danish: Native
		\item English: Fluent (Written and spoken)
	\end{itemize}

\section{Other Competences}
	\begin{itemize}
		%\item Embedded C, Arm Assembly, PCB design in Altium
		%\item Linux as a headless server
		%\item Python and bash for scripting and visualization.
		%\item Git for project management
		%\item VHDL and Chisel to better understand HW/SW integration.
		\item Programming (C, C++, 6502 Assembly, Arm Assembly, Bash, C\#, Java, Python, JavaScript, GLSL, F\#, Prolog, TeX)
		\item RTL Design (VHDL, Chisel)
		\item Circuit design
		\item Linux (As daily driver and headless)
		\item Git
		\item PCB design (Altium)
		\item Customer support
	\end{itemize}
	
\section{Links}
	\begin{itemize}
		\item \href{https://github.com/jondalnas}{GitHub}
		\item \href{https://github.com/jondalnas/CV}{GitHub CV}
		\item \href{https://www.linkedin.com/in/jonas-jensen-68a1091b1/}{LinkedIn}
	\end{itemize}
