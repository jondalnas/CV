% vim: setlocal spell spelllang=da
% LTeX: language=da_DK

\section{Profil}
	Jeg har en bachelor i Elektroteknologi og en kandidat i Informationsteknologi, begge fra DTU.
	Mine interesser ligger i et nært sammenspil mellem hardware- og indlejret software design, i at optimere og fejlfinding i eksisterende kode, og i at udforske Linux operativ systemet.
	Jeg har hele mit liv været interesseret i programmering og fejlfinding, da jeg som 12-årig begyndte at programmere i Java og lege med Arduino UNO'en.
	Lige siden den gang har min lyst til at udforske nye områder været enorm, hvor meget af min fritid går på små programmerings projekter i forskellige sprog eller vedligeholdelse af min personlige Linux server.
	Jeg er altid klar på en udfordring, hvis det betyder at jeg kan lære noget nyt, og imødekommer altid problemer med en interesse for udforskning.

\section{Erhvervserfaring}
	\begin{itemize}
		\item Feb 2025-Nu: Kasser hos K-Net
		\item Jun 2024-Nu: Daglig Driftsleder hos Nybrogård Netgruppe
		\item Jan 2023-Nu: 2024: Administrator hos Nybrogård Netgruppe
		\item Sep 2024-Dec 2024: Hjælpelære i 34365 IoT Prototyping hos DTU
		\item Jan 2024-Jun 2024: Hjælpelære i 34373 Introduction to microcontroller development for IoT using embedded C hos DTU
	\end{itemize}

\section{Uddannelse}
	\begin{itemize}
		\item 2023-2025: Cand. Polyt. Informationsteknologi, DTU
		\item 2020-2023: Bachelor i teknisk videnskab, BSc Eng. Elektroteknologi, DTU
		\item 2017-2020: Højere teknisk studentereksamen (HTX), H.C. Ørsted Gymnasiet Lyngby
	\end{itemize}

\section{Projekter}
	\begin{itemize}
		\item \textbf{Bachelor - \textit{Implementering af en ulineær differentialligning på en Raspberry Pi inklusive visualisering og I/O}}

			Projektet omhandlede at udvikle scripts til løsning af ulineære differentialligninger på en Raspberry Pi.
			I løbet af projektet lærte jeg Python, at optimere kode algoritmisk, og at slevstendigt stå som medleder på et projekt.

		\item \textbf{Kandidat - \textit{Development and implementation of embedded low-power tracking platform}}

			Projektet omhandlede at lave en low-power tracker, der skulle kunne leve på et enkelt CR2032 batteri i 10 år.
			I løbet af projektet lærte jeg at udvikle kode til low-power brug, at lave scripts til at hjælpe mig med at visualisere data, at designe PCB'er, og at lave universel C kode ved hjælp af define og ifdef.

		\item \textbf{Special kursus - \textit{Hardware acceleratorer til AI-baseret applikationer}}

			Projektet omhandlede at designe komponenter til AI i Chisel i følge ONNX standarden, hvorefter vi syntetiserede det på en FPGA.
			I løbet af projektet lærte jeg ONNX standarden for AI, at arbejde i en gruppe på hardware design, og at kunne komme i mål med et projekt uden behov for vejledning.

		\item \textbf{Personlige projekter}
			
			På min \href{https://github.com/jondalnas}{GitHub} kan største delen af mine personlige projekter i nyere tid findes.
			En tur af de mest relevante kan findes på mit \href{https://github.com/jondalnas/CV}{CV}.
	\end{itemize}

\section{Sprog}
	\begin{itemize}
		\item Dansk: Modersmål
		\item Engelsk: Flydende (Skrevet og talt)
	\end{itemize}

\section{Kompetencer}
	\begin{itemize}
		\item Programmering (C, C++, 6502 Assembly, Arm Assembly, Bash, C\#, Java, Python, JavaScript, GLSL, F\#, Prolog, TeX)
		\item RTL Design (VHDL, Chisel)
		\item Kredsløbsdesign
		\item Linux (Til daglig brug og headless)
		\item Git
		\item PCB design (Altium)
		\item Kundesupport
	\end{itemize}

\section{Links}
	\begin{itemize}
		\item \href{https://github.com/jondalnas}{GitHub}
		\item \href{https://github.com/jondalnas/CV}{GitHub CV}
		\item \href{https://www.linkedin.com/in/jonas-jensen-68a1091b1/}{LinkedIn}
	\end{itemize}
